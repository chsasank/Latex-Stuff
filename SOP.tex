\documentclass[a4paper]{article}
\usepackage{geometry}
 \geometry{
 a4paper,
 total={210mm,297mm},
 left=1.25in,
 right=1.25in,
 top=1.25in,
 bottom=1.25in,
 }
\usepackage{graphicx}
\usepackage{todonotes}
\begin{document}

\title{Research Statement / Statement of Purpose}
\author{Sasank}
\date{\today}
\maketitle
\begin{abstract}
Statement of Purpose should be a summary of your academic and professional goals, as well as a description of the research work you have done in your past, and what research work you plan to do as a graduate student at UT.
\end{abstract}


My first research experience is after my sophomore year when I had enough fundamentals to have an interest in research. 
%\todo{too specific? Too big may be}
I have worked, in team of three on a transform called Fractional Fourier Transform (FrFT).
Our problem was to give a proof/algorithm to estimate parameters of a given chirp signal. 
We have spent considerable time understanding the transform and how to discretize this transform for fast implementation. 
It turns out that discretizing this transform is not so straightforward as discretizing standard Fourier transform to DFT. There are many discretizations available and not all elegant properties of continuous FrFT are satisfied by each. 
We chose a discretisation which preserves properties relevant to our problem and still $O(n\log n)$ to compute. We then gave an algorithm in discrete domain using this discretisation and proved its correctness.

\todo{of course polish all this. think of this is as a outline.}
I've had a chance to attend and give a presentation in Indo-European winter academy on Optoelectronic emitters. Each participants did something.
I also got chance to attend CUHK school on Info Theory. This interested me the most.

I took a grad level course in Information and coding theory. 
I've done a independent project on implementing sum product algorithm for decoding LDPC codes and LZW compression algorithm.
While implementing later, I learnt a lot about error correcting codes and areas where this algorithm is relevant.
\todo{too cheesy?}

Many of my professors advised me to explore areas when I approached them. 
So, I took a grad level course on Processor design. We did ARM processor implementation. Made me learn a lot. However realised that this really isn't my area. Need math and all.


Intern in multilevel coded modulation for coherent optical modulation schemes.
I simulated the whole system in MATLAB. DSP part was interesting. \todo{really?}
Did expts on the whole system.
Couldn't publish because of cycle slips.

What I realised is experimenting with test algorithms is important for me. ML and DIP etc is good for that. Need not wait for equipment really for prototyping. This freedom is very refreshing.


BTP project on optimization. Currently stuck at proof.
Grad course on Stochastic optimization. Astounded by its applications. 
Paper review of deep learning theory. Learnt quite a bit about deep learning.
Concurrently image processing project choice. Random Forests.

Think applying deep neural nets to DIP and CV is promising area.
Handpicked features like HOG and SIFT etc. might not be the best.
Speech processing with handcrafted algos are going out of point.
The future is learning and automation.
What deep nets done for NLP for CV?.

\end{document}