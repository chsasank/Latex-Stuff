\documentclass[a4paper]{article}
\usepackage{geometry}
 \geometry{
 a4paper,
 total={210mm,297mm},
 left=1in,
 right=1in,
 top=1in,
 bottom=1in,
 }
\usepackage{enumitem}
\usepackage{amsmath}
\usepackage{amssymb}
\usepackage{amsthm}
\usepackage{tgpagella}
\usepackage[T1]{fontenc}
\usepackage{todonotes}
%% Define Theorem Stuff
% Theorem Styles
\newtheoremstyle{dotless}{}{}{\itshape}{}{\bfseries}{}{ }{}
\theoremstyle{dotless}
\newtheorem{theorem}{Theorem}[section]
\newtheorem{lemma}[theorem]{Lemma}
\newtheorem{proposition}[theorem]{Proposition}
\newtheorem{corollary}[theorem]{Corollary}
% Definition Styles
\newtheoremstyle{dotless}{}{}{}{}{\bfseries}{}{ }{}
\theoremstyle{dotless}
\newtheorem{defn}{Definition}[section]
\newtheorem{ex}{Example}[section]
\theoremstyle{remark}
\newtheorem{remark}{Remark}
%%


\begin{document}
\title{Error Correcting Codes Notes}
\author{Sasank}
\date{\today}
\maketitle


\section{Properties of Linear Block Codes}

\begin{itemize}
\item The minimum distance of a linear block code is equal to the
minimum weight of its nonzero codewords

\item Let C be a linear block code with parity check matrix H. There exists a codeword of weight w in C iff there exist w columns in H which sum to the zero vector.
 
\item \emph{Singleton Bound:} Let C be an $(n, k )$ binary block code with minimum distance
$d_{min}$.$$d_{min} \leq n - k + 1$$

Prove by puncturing first $d_min-1$ locations in each codeword and count number of codewords.

\item Let $A_i$ be the number of codewords of weight $i$ in C. Probability of undetected error over a BSC is given by $$P_{ue} = \sum_{i=1}^n A_ip^i(1-p)^{n-i}$$

\item \emph{Standard Array:} Rows are cosets of the code and first row in each row is called a coset leader. Any error pattern equal to a coset leader is correctable. So, every $(n,k)$ binary block code can correct $2^{n-k}$ error patterns.

\item \emph{Syndrome Decoding:} Each coset has a unique syndrome $y.H^T$. So, compute syndrome, find coset leader corresponding to that syndrome and add it to the received vector, $y$. 

\item Let $\alpha_i$ be the number of coset leaders of weight $i$ in C. Probability of decoding error over a BSC is given by $$P_e = 1- \sum_{i=0}^n\alpha_i p^i(1-p)^{n-i}$$

\item \emph{Hamming Bound}: Let C be an $(n, k )$ binary linear block code with minimum distance $d_{min} \geq 2t + 1$.
$$2^{n-k}\geq 1 + {n\choose 1}+{n\choose 2}+\dots+{n\choose t} $$
Prove by counting number of cosets. All patterns with weight less than or equal to t are coset leaders.

\item \emph{MacWilliams Identity:} Let $A_i$ be weight distribution of C and $B_i$ be that of $C^\perp$.
$$A(z) = 2^{-(n-k)}(1+z)^n B\left( \frac{1-z}{1+z}\right)$$
Can be useful in computing $P_{ue}$

\end{itemize}

\section{Examples of Linear Block Codes}
\subsection{Hamming Code}
For any integer $m \geq 3$, the code with parity check matrix consisting of all nonzero columns of length $m$ is a Hamming code. Some Properties:
\begin{itemize}
\item $n = 2^m - 1$
\item $k = 2^m - m -1$
\item $d_{min} = 3$
\end{itemize}

\subsection{Reed Muller Code}
Let $P(r,m)$ be the set of all boolean polynomials of $m$ variables having degree $r$ or less. Reed Muller code $RM(r,m)$ is given be the vectors \[\{v(f) | f \in P(r,m) \}\]
Where $v(f)$ is length $2^m$ vector containing values of $f$ evaluated at each of vector in $F_2^m$.
\begin{itemize}
\item Linear Code
\item n = $2^m$
\item k = $1 + {m\choose 1}+{m\choose 2}+\dots+{m\choose r}$
\item  \todo[inline]{read all this: decoding and min distance and all}
\end{itemize}


\section{Cyclic Code}
An $(n,k)$ linear block code C is a cyclic code if every cyclic shift of a codeword in C is also a codeword. Let $V(x)$ denote polynomial representation of V.
\subsection{Properties}
\begin{itemize}
\item Let $v^{(i)}(x)$ denote $i$th cyclic shift of $v(x)$. Then, $v^{(i)}(x) = x^iv(x)\mod{x^n+1}$
\item The nonzero code polynomial of minimum degree in a linear block code is unique. For $(n,k)$ cyclic code, constant term of such polynomial $g(x)$ is 1. We call $g(x)$ generator of the code
\item A binary polynomial of degree $n-1$ or less is a code polynomial if and only if it is a multiple of $g(x)$.
\item $\deg{g(x)} = n-k$
\item $g(x)$ generates a cyclic code iff $g(x)$ is a factor of $x^n + 1$.
\item \emph{Systematic encoding:} Divide $x^{n-k}u(x)$ by $g(x)$ to obtain reminder $b(x)$. Code polynomial is given by $b(x) + x^{n-k}u(x)$
\end{itemize}

-- Some Circuits here --

\subsection{Error Detection}
Syndrome polynomial $s(x) = r(x)\mod{g(x)}$
\begin{itemize}
\item If $x + 1$ is a factor of $g(x)$, all odd weight error patterns are detected
\item A polynomial over $F_2$ is said to be \textbf{irreducible} over $F_2$ if it has no factors other than 1 and itself. A degree $m$ irreducible polynomial is \textbf{primitive} if the smallest value of N for which it divides $x^n + 1$ is $2^m -1$
\end{itemize}

\section{Finite Groups}
\begin{defn} 
A set $G$ with binary operation $*$ defined on it is called a group if
\begin{enumerate}
\item $*$ is associative
\item There exists a identity element $e$, $a*e = e*a = a$
\item For every element $a$, there exists a inverse $b$, $a*b = b*a = e$
\end{enumerate}
Order of finite group is its cardinality.
\subsection{Some Definitons and Properties}
\begin{itemize}
\item \textbf{Cyclic group} $G = (g)$, for some element $g \in G$. It is called generator of $G$.
\item \textbf{Group isomorphism} is a bijection between two groups which `preserves' binary operation
\item Every cyclic group of order $n$ is isomorphic to $\mathbb{Z}_n$
\item A nonempty subset of $S$ of a group $G$ is called a \textbf{subgroup} of $G$ if for all $\alpha, \beta \in S$
		\begin{itemize}
		\item $\alpha + \beta \in S$
		\item $-\alpha \in S$
		\end{itemize}

\item If $S$ is a subgroup of a finite group $G$, then $|S|$ divides $|G|$. For any $g \in G$, the set $S+g$ = $\{s + g|s \in S\}$ is called a \textbf{coset} of S.
\item Every subgroup of a cyclic group is cyclic. There is a \emph{unique} subgroup for each divisor of order of the cyclic group.
\item A cyclic group of order $n$ has $\phi(n)$ generators where $\phi(n)$ is Euler's function. Can use this to prove
$$n = \sum_{d|n}\phi(d)$$
\end{itemize}
\end{defn}


\section{Finite Fields}
\begin{defn}
A set $F$ together with two binary operations $+$ and $*$ is a field if
\begin{enumerate}
\item $F$ is an abelian group under $+$ whose identity is called $0$
\item $F^* = F \setminus \{0\} $ is an abelian group under $*$ whose identity is called $1$
\item For any $a,b,c \in F$, $a * (b + c) = a * b + a * c$
\end{enumerate}
A finite field is a field with a finite cardinality.
\end{defn}
\subsection{Some Definitons and Properties}
\begin{itemize}
\item \textbf{Field isomorphism} is a bijection between two fields which `preserves' binary operations $+$ and $*$
\item Every field $F$ with a prime cardinality $p$ is isomorphic to $\mathbb{F}_p$. \textsl{(Prove this by observing that $F = (1)$)}
\item A nonempty subset of $S$ of a field $F$ is called a \textbf{subfield} of $F$ if for all $\alpha, \beta \in S$
		\begin{itemize}
		\item $\alpha + \beta \in S$
		\item $-\alpha \in S$
		\item $\alpha * \beta \in S\setminus\{0\}$
		\item $-\alpha^-1 \in S\setminus\{0\}$
		\end{itemize}

\item Let $F$ be a field with multiplicative identity $1$. The \textbf{characteristic} of $F$ is the smallest integer $p$ such that $1+1+1+\dots+1 \text{ (p times)} = 0$. The characteristic of a finite field is prime. \textsl{(If not, its divisors will be characteristic contradicting minimality)}
\item Every finite field has a prime subfield \textsl{($S = (1)$ is one such subfield)}
\item Any finite field has $p^m$ elements where $p$ is a prime and $m$ is a positive integer.
\textsl{(Let p be characterstic of $F$, observe that $F$ is a vector field over $\mathbb{F}_p$)}
\end{itemize}

\subsection{Polynomials over a Field}
\begin{defn}
A nonzero polynomial over a field $F$ is an expression $f (x ) = f_0 + f_1 x + f_2 x^2 +\dots + f_m x^m$ where $f_i \in F$ and $f_m \not= 0$. If $m =1$, $f(x)$ is said to be monic. 
The set of all polynomials over a field $F$ is denoted by $F[x]$.
\end{defn}

\begin{itemize}
\item A polynomial $a(x) \in F[x]$ is said to be a \textbf{divisor} of a polynomial $b(x) \in F[x]$ if $b(x) = q(x)a(x)$ for some $q(x) \in F[x]$. Trivial divisors are $\alpha$ and $\alpha f(x)$, $\alpha \in  F \setminus \{0\}$

\item An \textbf{irreducible polynomial} is a polynomial of degree $1$ or more which has only trivial divisors. A monic irreducible polynomial is called a \textbf{prime polynomial}.

\item Set of reminders when polynomials in $\mathbb{F}_p[x]$ are divided by a prime polynomial $g(x) \in \mathbb{F}_p[x]$ of degree $m$ is a field of order $p^m$.

\item Every monic polynomial $f(x) \in F[x]$ can be \emph{uniquely} written as a product of prime factors $a_i(x) \in F[x]$. 

\item If $f(x) \in F[x]$ has a degree $1$ factor $x - \alpha$ for some $\alpha \in F$ , then $\alpha$ is called a \textbf{root} of $f(x)$. $f(x)$ of degree $m$ can have at most $m$ roots.

\item In any field $F$, the multiplicative group $F^*$ of nonzero elements has at most one cyclic subgroup of any given order $n$. If it does, then its elements $\{ 1,\beta, \beta^2, \dots, \beta^{n-1}\}$ satisfy $$x^n-1 = (x-1)(x-\beta)(x-\beta^2)\dots(x-\beta^{n-1})$$

\item Elements of a finite field $F_q$ are $q$ distinct roots of $x^q-x$. \textsl{( $|(\beta)|$ divides $q-1$. So, $\beta^{(q-1)} = 1$ for all nonzero $\beta $ )}

\item $F_q^*$ is cyclic.
\todo{look at proof if time available}
\end{itemize}

\section{Minimal Polynomials}
Let $F_q$ be finite field with characteristic $p$. Thus, $F_q$ has a subfield isomorphic to $\mathbb{F}_p$. 
Consider polynomial $x^q - x \in F_q[x]$, it is also a polynomial in $F_p[x]$. Factorize $x^q - x$ into product of prime polynomials in $F_p[x]$
\[ x^q - x = \prod_i g_i(x) \]
$g_i(x)$ are called the \textbf{minimal polynomials} of $F_q$.

Since, $x^q - x = \prod_{\beta \in F_q}(x-\beta) = \prod_i g_i(x)$, 
$g_i(x) = \prod_{j=1}^{\deg{g_i(x)}}(x-\beta_{ij})$. So, each $\beta \in F_q$ is a root of exactly one minimal polynomial of $F_q$, called the minimal polynomial of $\beta$.

\begin{itemize}
\item Let $g(x)$ be the minimal polynomial of $\beta \in F_q$. $g(x)$ is the monic polynomial of least degree in $F_p[x]$ such that $g(\beta) = 0$. 
\textsl{( If $h(x)$ is such least degree polynomial, prove that it should divide $g(x)$. But $g(x)$ is prime polynomial. So $h(x) = g(x)$ ) }

\item For any $f(x) \in F_p(x)$, $f(\beta) = 0$ iff $g(x)$ divides $f(x)$ \textsl{ ( use previous result )}

\item For any $g(x) \in F_q(x)$, $g^p(x) = g(x^p)$ iff $g(x) \in F_p[x]$

\item Let $g(x)$ be the minimal polynomial of $\beta \in F_q$, If $q = p^m$, then the roots of $g(x)$ are of the form
\[
\beta, \beta^p, {\beta^p}^2, \dots , {\beta^p}^{n-1}
\]
where $n$ is a divisor of $m$.
\textsl{ ( Using previous result, if $y$ is a root, $y^p$ is also a root. If $n$ is smallest integer that ${\beta^p}^2 = \beta$, show that $n$ divides $m$ using the fact that ${\beta^p}^m = \beta$. Now show these can be only roots by invoking previous results.)} 
\end{itemize}


\section{BCH Codes}
\begin{defn}
Let $\alpha$ be a primitive element in $F_{2^m}$. The generator polynomial $g(x)$ of the t-error-correcting BCH code of length $2^m-1$ is the least degree polynomial in $\mathbb{F}_2[x]$ that has 
\[\alpha, \alpha^2, \alpha^3, \dots, \alpha^{2t}\] as its roots.
If $\phi_i(x)$ is minimal polynomial of $\alpha^i$, then $g(x)$ is LCM of $\phi_i(x), i = 1,2,\dots 2t$
\end{defn}

\begin{itemize}
\item For BCH code of parameters $m$ and $t$, we have
	\begin{itemize}
	\item $n-k\leq mt$
	\item $d_{min} \geq 2t+1$
	\end{itemize}

\item A degree $m$ irreducible polynomial in $F_2[x]$ is said to be primitive if the smallest value of $N$ for which it divides $X^N+1$ is $2^m-1$. The minimal polynomial of a primitive element is a primitive polynomial. \todo{How?}

\item Single error correcting BCH codes are Hamming Codes. \textsl{ ( $v(\alpha) = 0$ for code word $v$. Write $\alpha^i$ as a tuple)}

\item degree of generator polynomial $\deg{g(x)} \leq mt$. i.e, $n-k\leq mt$
\textsl{ ( Observe that even powers of $\alpha$ has same minimal polynomial as some odd power before it. Now, LCM of $m$ minimal polynomials $\leq mt$ )}

\item $d_{min} \geq 2t+1$ \todo{complete this.}
\end{itemize}

\end{document}
