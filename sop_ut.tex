\documentclass[11pt]{article}
\usepackage[letterpaper,margin=0.8in]{geometry}
\usepackage{xcolor}
\usepackage{fancyhdr}
\usepackage{tgschola} % or any other font package you like
\usepackage{todonotes}
\pagestyle{fancy}
\fancyhf{}
\fancyhead[C]{%
  \footnotesize\sffamily
  \yourname\quad
	\textcolor{blue}{\youremail}\quad
  \textcolor{blue}{\itshape\yourweb}\quad
  }

\newcommand{\soptitle}{Statement of Purpose}
\newcommand{\yourname}{Sasank Chilamkurthy}
\newcommand{\youremail}{\url{sasank@iitb.ac.in}}
\newcommand{\yourweb}{\url{http://home.iitb.ac.in/~sasank}}

\newcommand{\statement}[1]{\par\medskip
  \underline{\textcolor{blue}{\textbf{#1:}}}\space
}

\usepackage[
  colorlinks,
  breaklinks,
  pdftitle={\yourname - \soptitle},
  pdfauthor={\yourname},
  unicode
]{hyperref}


\begin{document}

\begin{center}
\LARGE\soptitle\\
\large of \yourname\ (ECE MS applicant for Fall -- 2015)
\end{center}

\hrule
\vspace{1pt}
\hrule height 1pt

\bigskip
I am applying for Electrical and Computer Engineering graduate program at University of Texas, Austin. 
I'm finishing a Bachelor's degree in Electrical Engineering at Indian Institute of Technology, Bombay in Spring 2015 with a minor in Mathematics. 
I am interested in signal processing and information theory.
\statement{Education}
I had always been interested in mathematics because of the relative ease with which it can model complex real world phenomenon. 
I chose Electrical Engineering as my major because of its strong mathematical flavour. 
As a freshman, I was fascinated at the rigour with which mathematics is approached. 
This, later, led me to do several graduate level courses from mathematics department earning a minor degree in Mathematics. 

The undergraduate program at IIT Bombay has exposed me to wide variety of topics in the vast field of Electrical Engineering. 
Signal processing and communication systems particularly appealed to me as mathematics applied to solve the practical problems of processing and transmitting data generated from various sources. 
It amazed me that seemingly abstract techniques like Fourier transform, convolution and optimization can be used to model and solve such a wide variety of problems ranging from image processing, speech recognition to communication systems.
Mathematical foundations of communications as formalised in information theory excited me. 
For a course project, I implemented decoding of LDPC codes using sum product algorithm on factor graphs. 
This made me appreciate some of the algorithms used in the modern communication systems. 
In my junior year, I attended an interesting winter school on information theory held at Institute of Network Coding, Chinese University of Hong Kong. 
This further cemented my interest in this field and prompted me to take graduate level electives like Information and Coding Theory, Digital Image processing and Stochastic Optimization.

\statement{Research}
Coming to my research experience, I worked on Fractional Fourier Transform (FrFT) in the summer of my sophomore year.
The problem was to estimate chirp signal parameters from very noisy samples using FrFT. This is of particular interest for sonar systems and acoustic communications. 
I surveyed available discretizations of the transform and chose one which is relevant to the problem and still as fast as FFT. 
I devised an algorithm which uses this discretization to estimate parameters and proved it using a clever trick of seeing given samples as a rescaled signal sampled in a specific way for which we know that discrete FrFT is an approximation of continuous FrFT. 
I also looked at a generalized version of FrFT which can handle higher order chirps like quadratic chirps. 
I proved uncertainty principle for this transform and worked on proving a tighter bound for real valued signals. 
In a nutshell, I understood mathematical foundations of a novel signal processing technique and used them to solve the problem at hand. This project won me an Undergraduate Research Award.

In the summer of my junior year, I interned at Centre for Optics, Photonics and Lasers (COPL), Quebec under supervision of Prof. Leslie Rusch. I worked on Coded Modulation for Coherent Optical Communication systems. 
My primary assignment was to apply MLCM heuristics proposed by Ramtin et al\footnote{ Ramtin Farhoudi and Leslie Ann Rusch, Multi-Level Coded Modulation for 16-ary Constellations in Presence of Phase Noise, Journal Of Lightwave Technology, Vol. 32, No. 6, March 15, 2014}
to various signal constellations. 
In particular, I had to compare, both by simulation and experiments, a constellation optimized for high phase noise and limited effective number of bits (ENOB) of transmitter with other constellations proposed in the literature. 
I modelled limited ENOB effects of DAC at higher Baud rates and simulated the coherent modulation setup with MLCM for various line-width lasers and signal constellations.
 My simulations showed that ENOB optimized constellation is the best performing constellation for high phase noise regimes. 
At low phase noises, however, square 16 QAM is the best performer. Next step was to validate this experimentally by calculating post FEC BER from offline processed data. 
Three different lasers of line-widths 10 KHz, 100 KHz and 1 MHz were used. For the first two lasers, experimental results echoed my simulations. 
It turned out that data from 1 MHz line-width laser has phases noise too large for my offline DSP to handle. 
In this project, I got hands-on experience on end to end optical communication system including advanced modulation techniques and different signal processing algorithms for carrier phase recovery.

For my B.Tech Project (senior thesis), I am working on water filling power allocation schemes for energy harvesting wireless networks. 
The problem is to find a water filling scheme which maximizes the expected average throughput in presence of both data arrival and energy recharge queues. 
I ran the simulations of the queues and they showed that average throughput is a quasi concave function of water level parameter of the water filling policy. 
I devised and implemented a stochastic approximation algorithm to maximize throughput and results matched with that of simulations. 
To prove convergence of this algorithm, it is necessary to show more properties of expected average throughput. I am working on this.

\statement{Teaching}
Apart from research experience, I have also gained considerable teaching experience at IIT Bombay. 
I have been teaching assistant for introductory mathematics courses of Calculus, Linear Algebra and Differential Equations. 
My duties included conducting weekly tutorials to help students understand the material and solve the problems from tutorial sheets and grading of answer sheets and assignments.
Thanks to the excellent feedback received from the students, my experience with teaching has built in me the confidence necessary to excel as a teacher.

\statement{Future Plans}
As a graduate student, I plan to work on signal processing, in particular image processing. 
I would like to solve problems related to perception of images. One such problem which excites me is automated perceptive image quality assessment.
Since humans are almost always the end users of image communications and compression, solving this problem will allow compression algorithms to directly optimise the perceptive quality of image rather than a proxy measure for it. 
I am also interested in exploring areas of machine learning and compressive sensing. 
My longterm career goal is to obtain a highly skilled professional position in industry and to contribute to cutting edge of the technology in the area of my choosing. 
I see excelling at graduate school as a natural and necessary intermediate step in this direction.

\statement{Why UTAsutin?}
ECE department at UTAustin is a well known centre of excellence. The department faculty consists of leading researchers engaged in frontline research of signal processing, communications and machine learning.
Interaction with such vigorous department will facilitate my development as an individual researcher.
Also, the research interests of the faculty are very compatible with those of mine. 
I find work done by Prof. Alan Bovik particularly interesting and would love to work with him. 
I also find work done by Prof. Constantine Caramanis and Prof. Alexandros G. Dimakis very interesting.
With guidance of these researchers, I am sure that I will make original contributions in my field and achieve my goals.

\medskip
In conclusion I would say that I am fully aware of kind of hard work, dedication and perseverance that is required for graduate research. Given my preparation and strong mathematical foundations,  I am confident that I have what it takes to be successful at a graduate school both in terms of subject background and motivation. I, therefore look forward to join UTAustin as a graduate student.

\end{document}