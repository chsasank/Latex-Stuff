\documentclass[11pt,a4paper,roman]{moderncv}        
\moderncvstyle{classic} 
\moderncvcolor{blue}                               % color options 'blue' (default), 'orange', 'green', 'red', 'purple', 'grey' and 'black'
%\renewcommand{\familydefault}{\sfdefault}         % to set the default font; use '\sfdefault' for the default sans serif font, '\rmdefault' for the default roman one, or any tex font name
%\nopagenumbers{}                                  % uncomment to suppress automatic page numbering for CVs longer than one page


% adjust the page margins
\usepackage[margin = 0.75 in]{geometry}
%\setlength{\hintscolumnwidth}{3cm}                % if you want to change the width of the column with the dates
\setlength{\makecvtitlenamewidth}{10cm}           % for the 'classic' style, if you want to force the width allocated to your name and avoid line breaks. be careful though, the length is normally calculated to avoid any overlap with your personal info; use this at your own typographical risks...

% personal data
\name{\Huge Sasank}{Chilamkurthy}
\address{\#134, Hostel 7, IIT Bombay, Mumbai, India}% optional, remove / comment the line if not wanted; the "postcode city" and "country" arguments can be omitted or provided empty
\phone[mobile]{+91 9892727514}                   % optional, remove / comment the line if not wanted; the optional "type" of the phone can be "mobile" (default), "fixed" or "fax"
\email{sasankchilamkurthy@gmail.com}                               % optional, remove / comment the line if not wanted
\homepage{home.iitb.ac.in/\textasciitilde sasank}                         % optional, remove / comment the line if not wanted

% to show numerical labels in the bibliography (default is to show no labels); only useful if you make citations in your resume
%\makeatletter
%\renewcommand*{\bibliographyitemlabel}{\@biblabel{\arabic{enumiv}}}
%\makeatother
%\renewcommand*{\bibliographyitemlabel}{[\arabic{enumiv}]}% CONSIDER REPLACING THE ABOVE BY THIS

% bibliography with mutiple entries
%\usepackage{multibib}
%\newcites{book,misc}{{Books},{Others}}
%----------------------------------------------------------------------------------
%            content
%----------------------------------------------------------------------------------
\begin{document}
%-----       resume       ---------------------------------------------------------
\makecvtitle

\section{Education}
\cventry{2011--15}{Bachelor of Technology}{Indian Instition of Technology, Bombay}{Mumbai}{}{
Major : Electrical Engineering\\ 
Minor : Mathematics \\ 
CGPA : 9.54/10
}  % arguments 3 to 6 can be left empty


\section{Research}
\cventry{Autumn 2014}{Scheduling for Energy Harvesting Wireless Networks}{\newline Guide: Prof. Abhay Karandikar, IIT Bombay}{}{}{
Consider a point to point link which is being charged by a renewable energy source where the amount of recharging energy entering the system varies inversely with the amount of energy available. How do we allocate power so as to maximize througput of such a link?
To reduce the search space, we restrict ourselves to waterfilling allocation schemes.
\begin{itemize}
\item Simulated the link and they show that average expected throughput over slots admit a quasi concave relation with waterlevel parameter of waterfilling policy.
\item Proposed a Stochastic Approximation algorithm based on infinitesimal perturbation analysis to maximize throughput.
\end{itemize}
}

\cventry{Summer 2014}{Coded Modulation for Coherent Optical Communication Systems}{\newline Guide: Prof. LA Rusch, Centre d'optique,photonique et laser (COPL), Quebec}{}{}{
To combat high phase noise from lasers in coherent optical communication systems, we need to use specialized signal constellations and coding schemes. How do various such phase noise optimized constellations stack up when used along with proposed Multi-Level Coded Modulation (MLCM) heuristics?
\begin{itemize}
\item Simulated 16 QAM coherent modulation for optical communication system and obtained BER vs OSNR curves for various lasers and signal constellations
\item Collected raw data from back-to-back experiments and extracted phase data from raw data with offline carrier Phase Recovery DSP algorithms
\item Analysed this data to evaluate the coding gain of MLCM
for various constellations
\end{itemize}
}


\cventry{Summer 2013}{Fractional Fourier Transform and Chirp Parameter Estimation}{\newline Guide: Prof. V.M.Gadre, IIT Bombay}{}{}{
Given samples of a noisy chirp signal, can we use a novel signal processing technique called Fractional Fourier Transform to determine the parameters of underlying chirp?
\begin{itemize}\itemsep -1pt
\item Surveyed literature on Fractional Fourier Transform and on various ways of discretizing it
\item Formulated and proved correctness of a DSP algorithm to estimate chirp parameters from noisy samples.
Evaluated accuracy of the algorithm in presence of the noise by simulation
\item Proved Uncertainity Principle for a new generalised transform extending fractional Fourier transform
\end{itemize}
}

\section{Awards and Achievements}
\cvlistitem{\textbf{Undergraduate Research Award} for work on Fractional Fourier Transform}
\cvlistitem{\textbf{Institute Academic Prize} for academic performance in the year 2013-14}
\cvlistitem{\textbf{28th rank} in India in IIT JEE 2011 exam taken by more than 500,000 students}
\cvlistitem{\textbf{3rd rank} in EAMCET 2011 exam written by 300,000 students}
\cvlistitem{\textbf{Gold medal} in Indian National \textbf{Chemistry Olympiad}}
\cvlistitem{\textbf{Top 300} in the country to be selected for Indian national \textbf{Physics} olympiad and Indian national \textbf{Astronomy} olympiad}


\section{Representative Projects}
\cventry{Autumn\\ 2014}{Traffic Sign Recognition}{\newline Guide: Prof. Ajit Rajwade, IIT Bombay}{}{}{
\begin{itemize}
\item Trained classifier using challenging German Traffic Sign Recognition Database
\item Implemented Random Forests, Linear Discriminant Analysis and Fisher's Linear Discriminant on Histogram of Oriented Gradients (HOG) featureset to achieve test set classification accuracy of about $97\%
$
\end{itemize}}

\cventry{Spring\\ 2014}{LZW compression algorithm and decoding LDPC codes}{\newline Guide: Prof. Ganesh Ramakrishnan, IIT Bombay}{}{}{
\begin{itemize}
\item Programmed Lempel-Zev-Welch compression and decompression algorithm 
\item Achieved about 50 \% compression ratio in the compressing large text files
\item Implemented decoding of LDPC codes using sum-product algorithm in Java using specially designed data structure: Factor Graph
\end{itemize}
}

\cventry{Spring\\ 2014}{Pipelined ARM Processor}{\newline Guide: Prof. Virendra Singh, IIT Bombay}{}{}{
\begin{itemize}
\item Architected a 6-stage pipelined processor based on the ARM7TDMI Instruction Set
\item Simulated the execution of instructions after designing the processor using Verilog HDL
\end{itemize}}

\cventry{Autumn 2013}{Wireless Communication using Amplitude Shift Keying (ASK)}{\newline Guide: Prof. J.Mukherjee, IIT Bombay}{}{}{
\begin{itemize}
\item Designed Analog circuits for ASK modulation and demodulation for medium wave band
\item Transmitted and received the modulated waveforms through monopole antennae
\item Used a microcontroller to send bit data and another to receive the data using UART protocol over this channel, thereby transmitting a text message wirelessly
\end{itemize}
}

\section{Seminars and Workshops attended}
\cvitem{December\\ 2013}{\textbf{Indo-European Winter Academy, IIT Guwahati} : One of 5 selected to represent IIT Bombay. Presented a 1-hour seminar on Optoelectronic Emitters covering physical principles and devices}

\cvitem{January\\ 2014}{\textbf{ITCSC-INC Winter School 2014, CUHK, Hong Kong} : One of 14 invited from India. Topics covered include Information Theory and Fourier transform for binary functions}

\section{Teaching Experience}
\cvitem{}{Tutored 40 strength class once a week helping them solve tutorial sheets and clarifying doubts among other duties like scrutinizing homeworks and answer sheets.}
\cvitemwithcomment{MA 105}{Calculus}{Autumn 2012,13,14}
\cvitemwithcomment{MA 106}{Linear Algebra}{Spring 2014}
\cvitemwithcomment{MA 108}{Differential equations}{Spring 2014}

\section{Key Courses}
\cvitem{Electical Engineering}{Information Theory, Error Correcting Codes, Digital Signal Processing, Digital Communications, Stochastic Optimisation, Probability and Random Processes}
\cvitem{Mathematics \& CS}{Image Processing, Data structures \& Algorithms, Quantum Information and Computing, Graph Theory, Real Analysis, Complex Analysis, Abstract Algebra}


\section{Extra-Curricular Activities}
\cvlistitem{Joint Secretary for Electrical Engineers Student Association (EESA), IIT Bombay. Planned and successfully executed two outings for 300 students}
\cvlistitem{Publicity Manager for Aagomani 2013, Annual festival of EE Department. Increased outreach of events leading to increase in footfall by 200\%}
\cvlistitem{Awarded \emph{bien} grade in Basic French course attested by Alliance Francaise de Bombay}
%\cvlistitem{Attended the Annual Training Camp organised by 2, MAH Regiment NCC and passed the B certificate examination under the authority of Ministry of Defence, India}
\end{document}
